\chapter{Informe de tutorías} \label{ch:meetings}
\section{1º Reunión (02/02/22): planteamiento general del TFG}
El proyecto consistirá en dos piezas fundamentales: una \textit{Aplicación Android} y un 
\textit{Servidor Web}. En este proyecto nos vamos a centrar en el Servidor Web, aunque en
fases posteriores será preciso hablar un poco de la aplicación android para el establecimiento
de la comunicación entre la aplicación y el servidor, ya que será necesario un mutuo acuerdo
entre ambas partes.

    \subsection{Servidor Web}

        \begin{enumerate}
            \item Almacenamiento de información en el formato correspondiente.
            \item Introducción de información detallada (ficha completa).
            \item Posibilidad de edición de las imágenes tomadas (marcar dentro de las
            fotografías las zonas de interés).
            \item Posibilidad de crear un pdf con la información contenida en la base de
            datos.
        \end{enumerate}

    \subsection{Fichas para la recogida de información}
    Para la recogida de datos en arqueología se hace uso de una serie de fichas donde podemos
    distinguir dos:

        \begin{enumerate}
            \item Registro de campos: esta ficha está dividida en una ficha de datos,
            conteniendo información básica de la pieza estratigráfica y una ficha fotográfica,
            que contiene las imágenes que describen dicha unidad.
            \item Ficha completa: en dicha ficha se hace un informe más exhaustivo completando
            la ficha de registro de campos.
        \end{enumerate}

    \subsection{Aspectos a tener en cuenta}
    Para la realización de este proceso es necesario aplicar la Ingeniería del Software,
    realizando los \textit{diagramas de actividad} y \textit{diagramas relacionales} necesarios.
    Además sería interesante incluir un manual de uso para el usuario.

\section{2º Reunión (15/02/22): jerarquización de los elementos}
En esta segunda entrevista que tuvo lugar en la ETSIIT con el tutor Daniel Sánchez Fernández
y el arqueólogo Francisco Javier Brao Gonzalez se hablaron distintos temas, sobre todo de cara
al diseño de la base de datos, ya que pudimos deducir lo siguiente:

    \begin{enumerate}
        \item La aplicación web necesitaría una base de datos relacional.
        \item Las unidades estratigráficas no siempre están asociadas a un hecho, por lo
        que habría que tener en cuenta esto para el diseño de la base de datos.
        \item Cada excavación está identificada con un número, a partir de este número
        se asignan los sucesivos.
        \item Aunque las excavaciones se identifiquen con un número, nosotros trabajaremos
        con un identificador interno.
    \end{enumerate}

Además de esto, se habló de la toma de imágenes y del marcado de puntos en las mismas en el
registro de campos (\textit{Croquis Planta} y \textit{Croquis sección}). Para ello, la
aplicación android tendría que enviar a la aplicación web los puntos de la imagen junto con
la imagen, luego, desde la aplicación web se tendrían que poder modificar dichos puntos mediante
alguna interfaz. 

Quedó pendiente por parte del arqueólogo la realización de un template de un posible informe
que se podría generar de una excavación. Para dicha generación, se cogerían los datos
que hay almacenados en la base de datos y se generaría un archivo word o un archivo similar,
que sea posteriormente modificable por el/la arqueólogo/a.


\section{3º Reunión (01/03/22): Diseño del modelo Entidad/Relación}
Una parte muy importante de la aplicación es la correcta modelización de la base de datos.
Para ello, \textit{Joaquín García Venegas} y yo hicimos un diseño inicial del modelo
E/R.

    \subsection{Entidades del modelo E/R}

        \begin{itemize}
            \item \textbf{Excavación}: es una entidad simple pero importante, ya que definirá
            la numeración (identificación) del resto de entidades del esquema 
            \item \textbf{Estancia}: compuesto por la asociación de hechos.
            \item \textbf{Hecho}: formado por un conjunto de unidades estratigráficas, que
            bien podrán ser tanto unidades estratigráficas sedimentarias como construidas 
            \item \textbf{Unidad estratigráfica}: es una de las entidades fundamentales de
            la base de datos, ya que es el nodo hoja en el árbol de la jerarquía.
            \item \textbf{Sedimentaria}: es una entidad que hereda de unidad estratigráfica.
            \item \textbf{Construida}: es una entidad que hereda de unidad estratigráfica.
            \item \textbf{Fotografía}: la poseen tanto las unidades estratigráficas como las
            estancias, serán las fotos descriptivas de una excavación
            \item \textbf{Inclusión}: entidad débil que surge como consecuencia de que una
            unidad estratigráfica de tipo sedimentaria puede contener restos de distintos
            componentes, haciendo necesaria la definición de una nueva entidad al modelo
            para poder reflejar una relación \textbf{muchos a uno entre Inclusión y
            Sedimentaria)}.
            \item \textbf{Material}: al igual que en la entidad anterior, surge con la
            necesidad de poder representar una relación \textbf{muchos a uno entre Material
            y Sedimentaria)}, ya que una unidad estratigráfica de tipo sedimentaria puede
            estar formada por muestras, metal, piedra, arcilla, etc.
        \end{itemize}


    \subsection{Dudas sobre el modelo} \label{subsec: doubts}
    Tras realizar el modelo, surgieron una serie de dudas acerca de las posibles relacionales
    y los datos de las fichas(tipos de datos y significado), lo cuál es normal ya que no
    dominamos los términos que se usan en arqueología con precisión. Las dudas que surgieron
    fueron las siguientes:

        \begin{enumerate}
            \item ¿Es el mismo material el de la ficha reducida que el de las unidades
            construidas?
            \item En la ficha de las estancias, no sabemos qué es el primer recuadro para
            dibujar.
            \item Problema de identificación de las estancias: en el pdf de metodologías, 
            se dice que se codifica con la palabra Estancia y tres dígitos (el primero
            indica la zona y los dos siguientes el nº de sector). Aparece un nº de sector,
            ¿es éste el nº de orden?
            \item ¿Son realmente importantes los atributos Plano nº, Seccion nº, Elevacion
            nº?
            \item ¿Una excavación puede pertenecer a varias UE?
            \item ¿El TPQ, TAQ y la fase en qué formato van, números, fechas, palabras, ..?
            \item ¿La ficha completa se rellena a partir de la ficha de campo?
            \item Las cotas, ¿van relacionadas con las pendientes?
        \end{enumerate}

    Evidentemente, estas dudas tendrán que ser resueltas por el arqueólogo Francisco Javier
    Brao Gonzalez en una reunión que se concertará posteriormente junto con el tutor,
    Daniel Sánchez Fernández.

\section{4º Reunión (07/03/22): Resolución de dudas}
En esta reunión se han resulto las dudas planteadas en la reunión anterior sobre el diseño
del modelo E/R. las correspondientes dudas junto a sus soluciones son las siguientes:

    \begin{enumerate}
        \item ¿Es el mismo material el de la ficha reducida que el de las unidades
        construidas? \textit{"No, el material de la ficha reducida corresponde a las
        unidades estratigráficas sedimentarias, mientras que el material de las
        construidas solo pertenece a éstas, además de ser único (es decir, una unidad
        construida no puede estar formada por distintos materiales)"}.
        \item En la ficha de las estancias, no sabemos qué es el primer recuadro para
        dibujar. \textit{"Este recuadro se corresponde con \textbf{la matriz de Harris}},
        dicha matriz se forma a partir de las UE en relación y los hechos en relación,
        colocando los distintos elementos correspondientemente en el recuadro".
        \item Problema de identificación de las estancias: en el pdf de metodologías, 
        se dice que se codifica con la palabra Estancia y tres dígitos (el primero
        indica la zona y los dos siguientes el nº de sector). Aparece un nº de sector,
        ¿es éste el nº de orden? \textit{"No, la zona y el sector siempre se rellenan,
        pero no forman parten de la identificación de la estancia. El nº de estancia
        no se forma a partir del númer de zona y el número de sector, un ejemplo podría
        ser por ejemplo con \textbf{ES001}, sin vinculación con el hecho"}.
        \item ¿Son realmente importantes los atributos Plano nº, Seccion nº, Elevacion
        nº? \textit{"Es simplemente documentación, son números que indican información
        de la Unidad Estratigráfica"}.
        \item ¿Una excavación puede pertenecer a varias UE? \textit{"No, normalmente una
        UE está asociada a una única excavación"}.
        \item ¿El TPQ, TAQ y la fase en qué formato van, números, fechas, palabras, ..?
        \textit{"El TPQ Y TAQ son números y la fase es un código ya definido que indica
        la época de la pieza (os tengo que pasar los códigos)"}.
        \item ¿La ficha completa se rellena a partir de la ficha de campo? \textit{"Sí,
        solo que hay que buscar la equivalencia entre algunos campos de la ficha
        reducida y la completa ya que cambian un poco"}.
        \item Las cotas, ¿van relacionadas con las pendientes? \textit{"Sí, la idea es
        que las zonas indiques la distancia que hay con respecto a un punto de referencia
        y luego las pendientes indican hacia donde está orientada la pieza: Norte-Sur,
        Suroeste-Este, etc, indican puntos de cardinalidad"}.
    \end{enumerate}

Tras esta reunión, se aclararon gran cantidad de dudas, lo que nos permitirá hacer un
correcto diseño del modelo E/R para su posterior paso a tablas y finalmente fusión, logrando
que el diseño de la base de datos sea correcto.

\section{5º Reunión (30/03/22): Corrección de memoria}
Tras el envío de la memoria de prácticas al tutor \textit{Daniel Sánchez Fernández}, se
han anotado una serie de recomendaciones a tener en cuenta en la memoria del proyecto. Entre
las anotaciones más relevantes destacan las siguientes:

    \begin{itemize}
        \item Es importante distinguir entre la elección de modelos en el diseño del problema
        y la elección de herramientas. Por ejemplo, elegir entre una base de datos relacional
        o noSQL sería una elección de diseño, mientras que elegir qué SGBD escoger (MySQL,
        SQLite, PostgreSQL, etc) sería una elección de herramientas.
        \item Incluir las citas desde las cuales se consulta la información.
        \item Comprobar que las imágenes usadas en la aplicación sean libres, o cambiarlas
        por imágenes de uso libre de repositorios como \href{https://pixabay.com/}{Pixabay} o
        \href{https://stock.adobe.com/es/}{Adobe Stock}.
        \item Dejar siempre claro la autoría de los archivos e imágenes incluidas en la memoria
        mediante una nota en la parte inferior de la misma o similar.
    \end{itemize}

Mencionadas estas anotaciones, se procede a modificar la estructura anterior de la memoria,
incluyendo un apartado nuevo dedicado al diseño. La sección de elección de herramientas
\ref{sec:tools} irá incluida en el capítulo de implementación, que recogerá la discusión de
herramientas elegidas y cómo se ha llevado a cabo la implementación de la aplicación.
