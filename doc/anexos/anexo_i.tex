\chapter{Informe de tutorías}
\section{1º Reunión (02/02/22): planteamiento general del TFG}
El proyecto consistirá en dos piezas fundamentales: una \textit{Aplicación Android} y un 
\textit{Servidor Web}. En este proyecto nos vamos a centrar en el Servidor Web, aunque en
fases posteriores será preciso hablar un poco de la aplicación android para el establecimiento
de la comunicación entre la aplicación y el servidor, ya que será necesario un mutuo acuerdo
entre ambas partes.

    \subsection{Servidor Web}

        \begin{enumerate}
            \item Almacenamiento de información en el formato correspondiente.
            \item Introducción de información detallada (ficha completa).
            \item Posibilidad de edición de las imágenes tomadas (marcar dentro de las
            fotografías las zonas de interés).
            \item Posibilidad de crear un pdf con la información contenida en la base de
            datos.
        \end{enumerate}

    \subsection{Fichas para la recogida de información}
    Para la recogida de datos en arqueología se hace uso de una serie de fichas donde podemos
    distinguir dos:

        \begin{enumerate}
            \item Registro de campos: esta ficha está dividida en una ficha de datos,
            conteniendo información básica de la pieza estratigráfica y una ficha fotográfica,
            que contiene las imágenes que describen dicha unidad.
            \item Ficha completa: en dicha ficha se hace un informe más exhaustivo completando
            la ficha de registro de campos.
        \end{enumerate}

    \subsection{Aspectos a tener en cuenta}
    Para la realización de este proceso es necesario aplicar la Ingeniería del Software,
    realizando los \textit{diagramas de actividad} y \textit{diagramas relacionales} necesarios.
    Además sería interesante incluir un manual de uso para el usuario.

\section{Glosario}
En este apartado se van a ir incorporando aquellas palabras o expresiones que son difíciles
de comprender, junto a su significado o algún comentario.

\textbf{Unidad estratigráfica (pieza estratigrática)}: volumen de roca con un origen
identificable y cuyo rango de edad se define por sus rasgos petrográficos, litológicos o
paleontológicos.