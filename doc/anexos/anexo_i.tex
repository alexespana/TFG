\chapter{Informe de tutorías}
\section{1º Reunión (02/02/22): planteamiento general del TFG}
El proyecto consistirá en dos piezas fundamentales: una \textit{Aplicación Android} y un 
\textit{Servidor Web}. En este proyecto nos vamos a centrar en el Servidor Web, aunque en
fases posteriores será preciso hablar un poco de la aplicación android para el establecimiento
de la comunicación entre la aplicación y el servidor, ya que será necesario un mutuo acuerdo
entre ambas partes.

    \subsection{Servidor Web}

        \begin{enumerate}
            \item Almacenamiento de información en el formato correspondiente.
            \item Introducción de información detallada (ficha completa).
            \item Posibilidad de edición de las imágenes tomadas (marcar dentro de las
            fotografías las zonas de interés).
            \item Posibilidad de crear un pdf con la información contenida en la base de
            datos.
        \end{enumerate}

    \subsection{Fichas para la recogida de información}
    Para la recogida de datos en arqueología se hace uso de una serie de fichas donde podemos
    distinguir dos:

        \begin{enumerate}
            \item Registro de campos: esta ficha está dividida en una ficha de datos,
            conteniendo información básica de la pieza estratigráfica y una ficha fotográfica,
            que contiene las imágenes que describen dicha unidad.
            \item Ficha completa: en dicha ficha se hace un informe más exhaustivo completando
            la ficha de registro de campos.
        \end{enumerate}

    \subsection{Aspectos a tener en cuenta}
    Para la realización de este proceso es necesario aplicar la Ingeniería del Software,
    realizando los \textit{diagramas de actividad} y \textit{diagramas relacionales} necesarios.
    Además sería interesante incluir un manual de uso para el usuario.

\section{2º Reunión (15/02/22): jerarquización de los elementos}
En esta segunda entrevista que tuvo lugar en la ETSIIT con el tutor Daniel Sánchez Fernández
y el arqueólogo Francisco Javier Brao Gonzalez se hablaron distintos temas, sobre todo de cara
al diseño de la base de datos, ya que pudimos deducir lo siguiente:

    \begin{enumerate}
        \item La aplicación web necesitaría una base de datos relacional.
        \item Las unidades estratigráficas no siempre están asociadas a un hecho, por lo
        que habría que tener en cuenta esto para el diseño de la base de datos.
        \item Cada excavación está identificada con un número, a partir de este número
        se asignan los sucesivos.
        \item Aunque las excavaciones se identifiquen con un número, nosotros trabajaremos
        con un identificador interno.
    \end{enumerate}

Además de esto, se habló de la toma de imágenes y del marcado de puntos en las mismas en el
registro de campos (\textit{Croquis Planta} y \textit{Croquis sección}). Para ello, la
aplicación android tendría que enviar a la aplicación web los puntos de la imagen junto con
la imagen, luego, desde la aplicación web se tendrían que poder modificar dichos puntos mediante
alguna interfaz. 

Quedó pendiente por parte del arqueólogo la realización de un template de un posible informe
que se podría generar de una excavación. Para dicha generación, se cogerían los datos
que hay almacenados en la base de datos y se generaría un archivo word o un archivo similar,
que sea posteriormente modificable por el/la arqueólogo/a.


\section{Glosario}
En este apartado se van a ir incorporando aquellas palabras o expresiones que son difíciles
de comprender, junto a su significado o algún comentario.

\textbf{Unidad estratigráfica (pieza estratigrática)}: volumen de roca con un origen
identificable y cuyo rango de edad se define por sus rasgos petrográficos, litológicos o
paleontológicos.