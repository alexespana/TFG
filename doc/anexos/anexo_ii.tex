\chapter{Manual de instalación}
En este manual de instalación, describiremos los pasos a seguir para poner en marcha la
aplicación web en una distribución Linux.


\section{Instalar Docker y Docker Compose}
Como hemos venido mencionando en el proyecto, para la ejecución de nuestra aplicación,
hemos usado \textbf{Docker} y \textbf{Docker Compose}, por lo que necesitaremos tener
ambos instalados en nuestro sistema. Para ello, se pueden seguir los siguientes tutoriales
de instalación, \href{https://docs.docker.com/engine/install/ubuntu/}{Docker} y
\href{https://docs.docker.com/compose/install/}{Docker Compose}, respectivamente. \\

\section{Definir variables de entorno}
Una vez completada la sección anterior, nos situamos sobre la rama de desarrollo
(\textit{development}) del proyecto, creamos un archivo llamado \textbf{.env} en
el directorio \textbf{code} y definimos las siguientes variables de entorno:
    \begin{enumerate}
        \item \textbf{DEBUG}: define si estamos en modo depuración o no. Como en este caso
        nos encontramos en un ambiente de desarrollo, deberemos darle valor \textbf{True}.

        \item \textbf{SECRET\_KEY}: define la clave secreta de nuestra aplicación. Realmente
        no es necesario darle valor, ya que se genera una clave aleatoria cada vez que se
        inicia la aplicación, sin embargo, es aconsejable darle un valor fijo, puesto que
        al generar una clave secreta cada vez, puede hacer que usuarios que estaban
        \textit{logueados} se \textit{deslogueen} de forma inesperada de la aplicación.

        \item \textbf{EMAIL\_HOST\_USER}: indica el correo electrónico a través del cuál la
        aplicación se conectará a través de SMTP para recibir las peticiones de registro y
        notificar a los usuarios cuando se les conceda permiso para \textit{loguearse} en
        el sistema.

        \item \textbf{EMAIL\_HOST\_PASSWORD}: clave de aplicación creada para permitir la
        conexión de Django con el correo electrónico. Será necesario realizar lo indicado
        en la sección \ref{subsubsec:send-mails} del proyecto.
    \end{enumerate}

\section{Crear imágenes y ejecutar contenedores}
Para la creación de las imágenes que necesita la aplicación, ejecutamos el siguiente comando
sobre el directorio \textbf{code}:

\begin{center}
    \textbf{docker-compose build}
\end{center}

Una vez finalizada la creación de las imágenes correspondientes (\textit{code\_web},
\textit{postgres} y \textit{adminer}), pasamos a ejecutar los contenedores mediante
el siguiente comando (también en el directorio code):

\begin{center}
    \textbf{docker-compose up}
\end{center}

Con este último comando tendríamos a los tres contenedores ejecutándose.

\section{Realizar migraciones y crear superusuario}
Finalmente, únicamente nos quedará realizar las migraciones de la base de datos. Para
ello, ejecutamos el siguiente comando:

\begin{center}
    \textbf{docker-compose run web python manage.py migrate}
\end{center}

Al ejecutar este comando, todas las tablas de la aplicación se crearán en la base de datos
y, además, los materiales predefinidos de las unidades estratigráficas se
añadirán. \\

Normalmente, al ejecutar la aplicación, necesitaremos crear un superusuario, puesto que se
utilizan permisos y roles en la aplicación. Si no tenemos ningún permiso no podremos
realizar ninguna acción sobre excavaciones, unidades, hechos, estancias, fotografías, etc.
Para ello, ejecutamos el siguiente comando y rellenamos los datos que se nos solicitan
(usuario, correo, contraseña y confirmación de contraseña):

\begin{center}
    \textbf{docker-compose run web python manage.py createsuperuser}
\end{center}

Con estos pasos, ya podríamos acceder a la aplicación a través del enlace
\url{http://localhost:8000/} o \url{http://127.0.0.1:8000/}. Además, podremos
\textit{loguearnos} con la cuenta de superusuario y realizar cualquier acción en la
aplicación.
