\chapter{Documentación adicional}
\section{Fichas de registro} \label{sec:registrationforms}
En este documento se sitúan todos los campos necesarios para la recogida de datos de campo
y de registro completo. Así, en él se reflejan las \textbf{UE reducidas}, \textbf{UE
estratigráficas de registro completo},  \textbf{hechos}, \textbf{estancias}, el
\textbf{inventario fotográfico}, \textbf{mobiliario}, \textbf{cerámica} y \textbf{tipología},
aunque estos tres últimos no los abarcaremos ene este proyecto.

Este documento se adjunta en primer lugar en la página siguiente.

\section{Metodología de registro}
En este documento, la parte principal y más importante describe \textbf{cómo se identifican
las entidades}, indicando el tipo de campos a los que se corresponden, enteros, palabras ya
previamente definidas como la letra identificativa de los Hechos junto con la numeración. Esta
información será de gran utilidad a la hora de elaborar el modelo E/R y tener claro qué
atributos forman la clave primaria y posibles claves candidatas de las entidades.

Este documento se adjunta en segundo lugar, después de \ref{sec:registrationforms}.

% Fichas de recogida de datos, tanto de campo como completas
\includepdf[pages=-]{info/FichasRegistroCampoCompleto.pdf}

% Jerarquía entre los elementos e identificación de ellos
\includepdf[pages=-]{info/MetodologiaFichasRegistro.pdf}

