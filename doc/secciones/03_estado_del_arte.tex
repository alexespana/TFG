\chapter{Estado del arte}
Hoy en día, el ritmo de desarrollo de las nuevas tecnologías crece exponencialmente y con
ello, las soluciones a las demandas que el mercado requiere en cada momento. En nuestro
caso la pregunta sería, ¿qué tecnología existe actualmente para la toma y gestión de datos
en excavaciones arqueológicas?\\

El hecho de que la profesión del arqueólogo sea un trabajo tan puramente práctico, quizás
le haya llevado a abstraerse en cierta medida del avance tecnológico, continuando con
métodos de recogida de datos que son poco prácticos o arcaicos. Por ejemplo, en la recogida
de datos se utilizan fichas de registro de los distintos elementos en papel, lo cual lo
hace un trabajo muy costoso y difícil de realizar. Como consecuencia de esto, en el mercado
existe un gran vacío en soluciones \textit{software} al respecto, siendo el número de
aplicaciones en el mercado muy bajo. En los apartados siguientes detallaremos cuáles son las
principales aplicaciones que existen en el mercado (privadas o de código libre)
que más se asemejarían a nuestra propuesta de aplicación.\\

\section{Piedrac}
Piedrac \cite{piedrac} es una aplicación de escritorio destinada a la documentación
arqueológica de campo. Permite registrar parte de la documentación que necesita una
excavación en una base de datos. Al igual que nuestra aplicación, permite recoger
información sobre unidades estratigráficas, muestras, coordenadas del contexto en el que
se realiza la excavación, imágenes, etc.\\

Además, se trata de una aplicación \textbf{portable}, siendo funcional en Windows, Linux
y MacOS, ya que la misma puede ejecutarse a partir de  un fichero \textbf{JAR}
\footnote{Tipo de archivo que permite \textbf{ejecutar} aplicaciones y herramientas escritas
en Java.} que no requiere ningún tipo de instalación por parte del usuario. El sistema
gestor de base de datos que utiliza es SQLite, conteniendo toda la base de datos en un
único fichero.\\

Uno de los puntos más fuertes de esta aplicación es que permite la exportación de distintos
tipos de datos en distintos formatos, que son los siguientes:

    \begin{itemize}
        \item Las tablas en formato \href{https://dev.socrata.com/docs/formats/csv.html}
        {\textbf{CSV} (Comma-separated Values)}.
        \item Las fichas de registro en \textbf{PDF} o en formato \textbf{HTML}.
        \item Imágenes a \textbf{JPG}.
        \item Coordenadas a \href{https://docs.fileformat.com/cad/dxf/}
        {\textbf{DXF} (Drawing Exchange Format)}.
    \end{itemize}

Finalmente, un punto importante de la misma es que se encuentra distribuida bajo la
licencia \textbf{GNU GPLv3} \cite{gplv3}, lo que permite obtener el código fuente de
la misma y modificarla como se requiera, siempre y cuando se redistribuya \textbf{bajo la
misma licencia}. Además, una de las grandes ventajas de este tipo de licencias es que
permiten que se lleve un \textbf{mantenimiento} adecuado de la aplicación y vaya
\textbf{mejorando} mediante contribuciones de desarrolladores interesados en participar en
el proyecto a lo largo de los años. \\


\section{ILIUM}
\textbf{ILIUM} \cite{ilium} es una aplicación Android destinada al registro de toda la
información arqueológica recogida durante el trabajo de campo, es decir, mientras se
excava. Al igual que la aplicación anterior, utiliza el sistema gestor de base de datos
SQLite, conteniendo toda la información de la excavación en la propia tableta o smartphone
que se ha utilizado.\\

Esta aplicación tiene consecuentemente como finalidad recoger los datos que pertenecen a
la ficha de registro de campo reducida, correspondiente al anexo \ref{sec:registrationforms},
por lo que recogería una parte de la finalidad para la que será creada nuestra aplicación web,
aunque, como es evidente, los contextos son muy distintos, pues nuestro proyecto consistirá
en la creación de una aplicación web, no Android. \\


\section{Sistemas de Información Geográfica}
Otra de las herramientas más utilizadas para la documentación arqueológica son los
\textbf{Sistemas de Información Geográfica (SIG)} \cite{gis}. Principalmente se utilizan
para localizar en el espacio el yacimiento y toda la información referente al mismo.
Básicamente consiste en una base de datos donde se va almacenando toda la información de una
excavación para posteriormente formar un mapa a partir de dicha información. \\

Al igual que las aplicaciones anteriores, los SIG sirven como base para la posterior
documentación de las excavaciones, pudiendo contener grandes cantidades de información
que, posteriormente será utilizada para documentar de forma espacial la excavación. Este
tipo de herramientas se utilizan a nivel de escritorio, es decir, son aplicaciones que
se instalan en un ordenador propio y funcionan sobre el mismo, al igual que Piedrac.\\

En la actualidad, el Sistema de Información Geográfica más utilizado por empresas es
\href{https://www.qgis.org/es/site/}{QGIS}. Es una herramienta que, al igual que Piedrac,
es portable entre distintos sistemas operativos (Windows, Linux, MacOS), y que es libre
y de código abierto, lo que hace que vaya evolucionando continuamente por la comunidad.

\section{MyFindings, nuestro proyecto}
Como hemos podido observar en los puntos anteriores, realmente esta disciplina no ha pasado
por una fase de digitalización, al menos hasta el momento, y las soluciones \textit{software}
que hay disponibles en el mercado, o no solucionan el problema o se quedan obsoletas. Este
hecho hace que nuestra aplicación tome un punto estratégico en el mercado, proporcionando al
cliente servicios nunca antes ofrecidos por otras aplicaciones. Por ejemplo, en las
aplicaciones anteriormente descritas, se han percibido las siguientes limitaciones:

    \begin{itemize}
        \item \textbf{Ejecución en local}: a pesar de ser aplicaciones portables entre
        distintos sistemas operativos (a excepción de la aplicación Android), que se
        ejecuten localmente puede derivar en distintos problemas como puede ser la pérdida
        de información por un fallo del mismo. Por ejemplo, que deje de funcionar el
        dispositivo, que se desinstale la aplicación por equivocación, que se borren los
        datos vinculados, etc.
        
        \item \textbf{Aplicaciones que no son multiusuario}: las aplicaciones que se
        ejecutan en el ordenador propio, no son multiusuario, es decir, no se puede
        acceder a la misma desde otro ordenador. Esto provoca que en una misma excavación
        no puedan participar varios arqueólogos simultáneamente, y si lo hacen, no tendrán
        la información de la excavación sincronizada, resultando en inconsistencias de
        información.
    \end{itemize}

Las dos limitaciones anteriores son muy importantes, ya que afectan directamente a la
escalabilidad de la aplicación. Myfindings solucionaría estos problemas, ya que tiene como
finalidad que múltiples usuarios puedan trabajar en una misma excavación y, además, ayudar a
la documentación de las mismas. En concreto, la idea es que presente las siguientes
características:

    \begin{itemize}
        \item Ayuda a la \textbf{documentación} de excavaciones arqueológicas.
        \item Aplicación web \textbf{multiusuario}, donde los usuarios podrán trabajar
        simultáneamente en varias excavaciones.
        \item \textbf{Generación automática de informes} a partir de la información 
        disponible en la base de datos.
        \item Open source, distribuido bajo la licencia \textbf{GNU GPLv3}, haciendo que
        el \textit{software} siga en mantenimiento e incluyendo nuevas funcionalidades por
        la comunidad.
    \end{itemize}

Con estas características ofreceríamos a los usuarios una forma única y nueva en el
mercado para la toma y gestión de datos en excavaciones.
