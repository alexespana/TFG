\chapter{Análisis del problema}
 
En todo proceso de Ingeniería del Software se necesita seguir un procedimiento muy metódico
y preciso, dividiendo el desarrollo del software en distintas fases que nos ayudarán a
conseguir un producto que se ajuste a las necesidades que requiere el problema.

\section{Ingeniería de requisitos}
Durante esta fase se cubrirán y proporcionarán las técnicas y los mecanismos apropiados
para:

    \begin{itemize}
        \item Analizar y entender las necesidades de los arqueólogos.
        \item Evaluar la viabilidad de las distintas propuestas (necesidades).
        \item Negociar un solución razonable/viable para ambas partes.
        \item Controlar y administrar los requisitos a lo largo del proceso de desarrollo.
    \end{itemize}

\subsection{Requisitos funcionales}
Describen la interacción entre la Aplicación Web y su entorno (base de datos, usuarios, 
comunicación con la App Android, etc), indicando cómo debe actuar frente a situaciones o
entradas determinadas.

    \begin{itemize}
        \item \textbf{RF-1}. Gestión de excavaciones.
            \begin{itemize}
                \item \textbf{RF-1.1}. Consultar excavación.
                \item \textbf{RF-1.2}. Añadir excavación.
                \item \textbf{RF-1.3}. Modificar excavación.
                \item \textbf{RF-1.4}. Eliminar excavación.       
                \item \textbf{RF-1.5}. Consultar unidades estratigráficas asociadas.            
            \end{itemize}
        \item \textbf{RF-2}. Gestión de estancias.
            \begin{itemize}
                \item \textbf{RF-2.1}. Consultar estancia.
                \item \textbf{RF-2.2}. Añadir estancia.
                \item \textbf{RF-2.3}. Modificar estancia.
                \item \textbf{RF-2.4}. Eliminar estancia.       
            \end{itemize}
        \item \textbf{RF-3}. Gestión de hechos.
            \begin{itemize}
                \item \textbf{RF-3.1}. Consultar hecho.
                \item \textbf{RF-3.2}. Añadir hecho.
                \item \textbf{RF-3.3}. Modificar hecho.
                \item \textbf{RF-3.4}. Eliminar hecho. 
                \item \textbf{RF-3.5}. Consultar unidades estratigráficas asociadas.          
            \end{itemize}
        \item \textbf{RF-4}. Gestión de unidades estratigráficas.
            \begin{itemize}
                \item \textbf{RF-4.1}. Consultar unidad estratigráfica.
                \item \textbf{RF-4.2}. Añadir unidad estratigráfica.
                \item \textbf{RF-4.3}. Modificar unidad estratigráfica.
                \item \textbf{RF-4.4}. Eliminar unidad estratigráfica.          
            \end{itemize}
        \item \textbf{RF-5}. Gestión de inventario fotográfico.
            \begin{itemize}
                \item \textbf{RF-5.1}. Consultar fotografía.
                \item \textbf{RF-5.2}. Añadir fotografía.
                \item \textbf{RF-5.3}. Modificar fotografía.
                \item \textbf{RF-5.4}. Eliminar fotografía.    
            \end{itemize}
        \item \textbf{RF-6}. Gestión de materiales.
            \begin{itemize}
                \item \textbf{RF-5.1}. Consultar material.
                \item \textbf{RF-5.2}. Añadir material.
                \item \textbf{RF-5.3}. Añadir material.
                \item \textbf{RF-5.4}. Eliminar material.    
            \end{itemize}
        \item \textbf{RF-7}. Gestión de usuarios.
            \begin{itemize}
                \item \textbf{RF-7.1}. Registro de usuario.
                \item \textbf{RF-7.2}. Alta de usuario (autorizado).
                \item \textbf{RF-7.3}. Modificar contraseña.
                \item \textbf{RF-7.4}. Recuperación de contraseña.
                \item \textbf{RF-7.4}. Dar de baja.
                \item \textbf{RF-7.5}. Avisos.
                    \begin{itemize}
                        \item \textbf{RF-7.5.1}. Enviar aviso en petición de registro.
                        \item \textbf{RF-7.5.2}. Enviar aviso en registro concedido.
                        \item \textbf{RF-7.5.3}. Enviar aviso en recuperación de contraseña.
                    \end{itemize}
            \end{itemize}
        \item \textbf{RF-8}. Gestión de permisos.
            \begin{itemize}
                \item \textbf{RF-8.1}. Consultar permiso de usuario.
                \item \textbf{RF-8.2}. Añadir permiso de usuario.
                \item \textbf{RF-8.3}. Modificar permiso de usuario.
                \item \textbf{RF-8.4}. Eliminar permiso de usuario.
                \item \textbf{RF-8.5}. Añadir grupo de permisos.
                \item \textbf{RF-8.6}. Modificar grupo de permisos.
                \item \textbf{RF-8.7}. Eliminar grupo de permisos.
            \end{itemize}
        \item \textbf{RF-9}. Generación de informes.
            \begin{itemize}
                \item \textbf{RF-9.1}. Informe completo de excavación.
            \end{itemize}
    \end{itemize}

\subsection{Requisitos no funcionales}
Este tipo de requisitos describen restricciones o cualidades de nuestra aplicación web que
no tienen una relación directa con las funcionalidades del sistema.

    \begin{table}[H]
        \centering
        \begin{tabular}{|l |l |} \hline

            \textbf{RNF-1} 
             & \underline{Eficiencia} \\
             & \tabitem El sistema tiene que tener un tiempo de respuesta corto.  \\
             & \tabitem El sistema tiene que ser capaz de almacenar gran cantidad de \\
             & excavaciones. \\ \hline

             \textbf{RNF-2} 
             & \underline{Usabilidad} \\
             & \tabitem Tiene que ser compatible con ordenadores, móviles y tablets.  \\
             & \tabitem Fácil de usar, se hará un uso intuitivo de forma sencilla. \\ \hline

             \textbf{RNF-3} 
             & \underline{Seguridad} \\
             & \tabitem Acceso al sistema con usuario y contraseña únicos.  \\
             & \tabitem Cada usuario verá únicamente su información privada. \\ \hline

             \textbf{RNF-4} 
             & \underline{Legalidad} \\
             & \tabitem Debe cumplir la legislación sobre la gestión de información. \\
             & confidencial \\ \hline

             \textbf{RNF-5} 
             & \underline{Disponibilidad} \\
             & \tabitem El sistema debe estar disponible el máximo tiempo posible. \\ \hline

             \textbf{RNF-6} 
             & \underline{Implementación} \\
             & \tabitem Debe ser accesible desde un dispositivo móvil.  \\
             & \tabitem Debe ser accesible desde un navegador web. \\ \hline

             \textbf{RNF-7} 
             & \underline{Operaciones} \\
             & \tabitem Si se requiriese alguna modificación, ésta la realizará el \\
             & administrador.  \\ \hline

             \textbf{RNF-8} 
             & \underline{Soporte} \\
             & \tabitem Durante el mantenimiento, el sistema estará inoperativo. \\ \hline

        \end{tabular}
        \caption{Tabla de requisitos no funcionales}
        \label{tab:rfn-table}
    \end{table}


\subsection{Requisitos de información}
Este tipo de requisitos describirán necesidades de almacenamiento de información en
la aplicación web.

\begin{table}[H]
    \centering
    \begin{tabular}{|l |l |} \hline

        \textbf{RI-1} 
         & \underline{Usuarios} \\
         & \tabitem Información necesaria de los usuarios a la hora de registrarse en la \\
         & aplicación.  \\
         & \tabitem \textbf{Contenido}: nombre, correo electrónico, contraseña. \\ 
         & \textbf{Requisitos asociados}: RF-7 \\ \hline

    \end{tabular}
    \caption{Tabla de requisitos de información}
    \label{tab:ri-table}
\end{table}
