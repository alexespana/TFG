\chapter{Análisis del problema}
 
En todo proceso de Ingeniería del Software se necesita seguir un procedimiento muy metódico
y preciso, dividiendo el desarrollo del software en distintas fases que nos ayudarán a
conseguir un producto que se ajuste a las necesidades que requiere el problema.

\section{Ingeniería de requisitos}
Durante esta fase se cubrirán y proporcionarán las técnicas y los mecanismos apropiados
para:

    \begin{itemize}
        \item Analizar y entender las necesidades de los arqueólogos.
        \item Evaluar la viabilidad de las distintas propuestas (necesidades).
        \item Negociar un solución razonable/viable para ambas partes.
        \item Controlar y administrar los requisitos a lo largo del proceso de desarrollo.
    \end{itemize}

\subsection{Requisitos funcionales}
Describen la interacción entre la Aplicación Web y su entorno (base de datos, usuarios, 
comunicación con la App Android, etc), indicando cómo debe actuar
frente a situaciones o entradas determinadas.

    \begin{itemize}
        \item 
    \end{itemize}

\subsection{Requisitos no funcionales}
Este tipo de requisitos describen restricciones o cualidades de nuestra aplicación web que
no tienen una relación directa con las funcionalidades del sistema.

    \begin{itemize}
        \item 
    \end{itemize}

\subsection{Requisitos de información}
Este tipo de requisitos describirán necesidades de almacenamiento de información en
la aplicación web.

    \begin{itemize}
        \item 
    \end{itemize}

\section{Tecnologías disponibles}
En la elección de cualquier herramienta en un proyecto software, se debe seguir un
proceso de selección de herramientas entre distintas alternativas, especificando claramente
los requisitos que deben cumplir para poder escoger aquella que mejor se adapte a las
necesidades del proyecto (en este caso a la aplicación web).

\subsection{Framework para desarrollo web}

    \subsubsection{Ruby on Rails}
    \textbf{Ruby on Rails} (también conocido como Rails) \cite{ruby-on-rails} es un framework
    para la creación de aplicaciones web escrito en Ruby. Sigue una arquitectura
    (Model-View-Controller) y diseño RESTfull. Está diseñado para facilitar la programación
    web al hacer una estimación de los componentes principales que se necesitan para comenzar.

    Este software asume que hay una mejor forma de hacer las cosas y está diseñado para
    ello, por lo que es un código muy optimizado y hecho para aumentar la productividad
    si se hace el desarrollo de la forma "The Rails Way".

    \subsubsection{Django}
    \textbf{Django} \cite{django} es el framework para Python más usado, está enfocado a sitios
    basados en bases de datos. Permite un desarrollo software muy rápido y limpio. Tiene una
    arquitectura MVT (Model-View-Template), donde el View sería el equivalente a Controller
    en MVC.

    Al igual que el anterior framework mencionado, Django ha sido diseñado por programadores
    experimentados, y se encarga de las partes más complejas del desarrollo web, permitiendo
    que el programador se centre en escribir la aplicación sin necesidad de preocuparse
    por otras cuestiones.

    \subsubsection{Laravel}
    \textbf{Laravel} \cite{laravel} es el mejor framework de PHP para desarrollar aplicaciones
    y servicios web con PHP5, PHP7 y PHP8. Al igual que Ruby on Rails, sigue una arquitectura 
    MVC, por lo que es sencillo relacionar las distintas componentes de la aplicación.

    Es un framework moderno que ofrece muchas utilidades a los desarrolladores y permiten un
    desarrollo ágil de las aplicaciones web. Además, pone mucho interés en la calidad del
    código, el mantenimiento y la escalabilidad.

    \subsubsection{Angular}
    \textbf{Angular} \cite{angular} es un framework desarrollado en TypeScript para el
    desarrollo de aplicaciones web. Está mantenido por Google, utilizado para crear y mantener
    aplicaciones SPA (Single Page Application), es decir páginas web que interaccionan con el
    usuario dinámicamente sobreescribiendo la página con información nueva.

    Al contrario que los frameworks anteriormente mencionados, Angular no usa un modelo MVC
    o MVT, sino que permite el desarrollo del front-end y el back-end de forma totalmente
    independiente.

\subsection{Framework CSS}

    \subsubsection{Tailwind CSS}
    \textbf{Tailwind CSS} \cite{tailwind-css} es un framework CSS que funciona escaneando los
    archivos HTML, componentes Javascript y otros templates generando los correspondientes
    estilos y escribiéndolos en un archivo CSS estático. No tiene muchos componentes, sino
    clases de utilidad que pueden aplicarse directamente sobre el código HTML del proyecto.
    Además de esto, permite una gran optimización del peso del código CSS mediante unos flujos
    de desarrollo.

        
    \subsubsection{Bootstrap}
    \textbf{Bootstrap} \cite{bootstrap} es uno de los frameworks más populares y usados para
    el desarrollo de páginas web. Fue desarrollado por Twitter en 2010 para uso de la compañía
    pero más tarde pasó a ser código abierto. Este framework combina CSS y Javascript. Entre
    sus características fundamentales podemos encontrar:
    
    % \begin{list}{\textbullet}{ 
        %     \addtolength{\labelsep}{1mm}    % Distancia hasta la etiqueta
        %     \addtolength{\itemsep}{-2mm}    % Separación entre items
        %     \setlength{\itemindent}{5mm}}   % Identación de los items
        % \end{list}
        
        \begin{itemize}
            \item \textbf{Responsive design}: adaptación automática a distintos dispositivos
            como móviles y tablets.
            \item \textbf{Grid System}: posicionamiento de elementos en la página
            \item \textbf{Interface UI}: incluye formularios, botones, menús, etc.
        \end{itemize}
        
        \subsubsection{Foundation}
        \textbf{Foundation} \cite{foundation} es el principal competidor de Bootstrap, es un
        framework orientado al desarrollo de sitios web totalmente adaptativos bajo el enfoque
        \textit{mobile first}, que es una metodología de desarrollo donde se tiene en cuenta
        en primer lugar los dispositivos móviles. Esta metodología es totalmente contraria a
        la que usa Bootstrap ya que en él se usa \textit{Responsive Web Design} y en éste
        \textit{Mobile First Web Design}.
        
        Además de su metodología de desarrollo, posee ciertas características muy interesantes:
        
        \begin{itemize}
            \item \textbf{Fastclick}: eliminación del retraso al pulsar en dispositivos
            móviles.
            \item \textbf{Semántico}: todo es semántico, permitiendo tener un lenguaje de
            marcado limpio sin sacrificar su funcionalidad y velocidad.
            \item \textbf{Off Canvas}: creación de menús dinámicos
            \item \textbf{Customizable}: permite hacer diseño completamente personalizables,
            pudiendo eliminar elementos, definir el tamaño de las columnas, colores,
            fondos, etc
        \end{itemize}

    \subsubsection{Materialize CSS}
    \textbf{Materialize CSS} \cite{materialize-css} sigue el principio \textit{Material Design},
    es decir, ofrece componentes ya listos para ser utilizados, por supuesto adaptativo.
    Además, integra comportamientos dinámicos mediante JavaScript y no necesita de Jquery para
    funcionar.

    Entre sus principales características se encuentran:
        
        \begin{itemize}
            \item Al igual que otros frameworks mencionados, permite un diseño adaptativo.
            \item Permite crear menús laterales desplegables o abiertos según la resolución
            del dispositivo.
            \item Diseños utilizando la filosofía Material Design (colecciones, tarjetas,
            barras de navegación, modales, toast, etc).
            \item Añade utilidades como \textbf{Parallax} (técnica de diseño web en la que se crea
            un efecto de profundidad al hacer scroll)
        \end{itemize}

\subsection{Base de Datos}
Antes de pasar a la elección de la Base de Datos concreta necesitamos especificar si
necesitamos una base de datos SQL o una base de datos noSQL. 
\begin{itemize}
    \item \textbf{Bases de Datos relacionales}: son aquellas que utilizan el lenguaje SQL (
    \textit{Structure Query Languaje}), un lenguaje de consulta estructurado. Este tipo de 
    bases de datos están formadas por filas y columnas. En cada una de las filas residen
    registros, mientras que las columnas corresponden a campos de los mismos. Son el tipo
    de bases de datos más estándarizado y con más uso actualmente.
    \item \textbf{Base de Datos noSQL}:  son aquellas bases de datos diseñadas para permitir
    grandes cantidades de datos, tipos de datos complejos, más índices,distintos tipos de
    consulta, etc. Este tipo de base de datos surgió para dar solución a los problemas de
    escalabilidad y rendimiento de las bases de datos relacionales en las que hay miles de
    usuarios concurrentemente y realizando millones de consultas.
\end{itemize}

En la siguiente tabla, se realiza una breve comparación entre ambos tipos: 

    \begin{table}[H]
        \begin{center}
            \begin{tabular}{ |l|l| } \hline
                \textbf{SQL} & \textbf{noSQL} \\ \hline
                Tablas & Hashes, listas \\
                Escalabilidad vertical & Escalabilidad horizontal \\
                Datos estructurados & Datos menos estructurados \\ 
                Consistencia & Dinamicidad \\ \hline
            \end{tabular}
            \caption{tabla con la comparativa entre SQL y noSQL}
            \label{tab:databases}
        \end{center}
    \end{table}

Dada la naturaleza de nuestro problema, donde existe una clara jerarquización de los
elementos con los que se trabaja: \textbf{UE, sector, año, hecho, estructura}, ordenados
de menor a mayor complejidad, intuimos que necesitamos hacer uso de un sistema de
almacenamiento de información que cumpla con el esquema entidad-relación, por lo tanto,
deberemos usar una base de datos SQL. Los distintos SGBD (Sistemas Gestores de Bases de
Datos) datos que podemos encontrar son los siguientes:

    \subsubsection{MySQL}
    \textbf{MySQL} \cite{mysql} es un sistema de gestión de bases de datos relacionales de
    código abierto (también conocido como RDBMS, Relational Database Management System). Es
    una de las bases de datos más populares en la actualidad y usa un modelo
    cliente-servidor.
    

    \subsubsection{SQLite}
    \textbf{SQLite} \cite{sqlite} es un RDBMS de tamaño reducido (aproximadamente 275 kiB)
    que a diferencia de otros RDBMS como mySQL, no usa un proceso cliente y un proceso
    servidor(modelo cliente-servidor). En este caso, funciona como un servidor
    propio independiente, haciendo uso de llamadas a subrutinas y funciones integradas en el
    propio código fuente que eliminan la necesidad de hacer consultas entre procesos
    separados.

    \subsubsection{PostgreSQL}
    \textbf{PostgreSQL} \cite{postgresql} es un sistema de gestión de base de datos relacional
    que está orientado a objetos, de código abierto y además gratuito. Este RDBMS posee tipos
    de datos avanzados y permite optimizar de forma considerable el rendimiento, este tipo de
    características por lo general solo se ofrecen en sistemas de bases de datos privativas.
    Además de las características anteriormente mencionadas, permite un control de grandes
    volúmenes de datos y tiene soporte completo para \textbf{ACID} (\textit{Atomicity,
    Consistency, Isolation, Durability}).

\subsection{Solución propuesta}
Como se ha mencionado anteriormente, en cualquier elección de herramientas se debe señalar
claramente las características que debe cumplir para así poder descartar con facilidad
entre las disintas alternativas.

    \subsubsection{Elección del Framework de Desarrollo Web}
    El framework de Desarrollo Web debe cumplir con una serie de \textbf{requisitos} para
    su aceptación:

        \begin{enumerate}
            \item Que haga una deducción de los elementos principales de la aplicación,
            permitiendo centrarnos exclusivamente en el desarrollo de la misma.
            \item Que pueda permitir la configuración del mismo con distintas bases de
            datos.
            \item Que posea un sistema de gestión de usuarios (\textbf{autenticación}).
            \item Que el manejo de los modelos de la base de datos pueda abstraerse
            independientemente de la base de datos elegida mediante un \textbf{Object
            Relational Mapping} (ORM).
            \item Que incluya una fácil creación de la API para la posible comunicación
            con la App android.
        \end{enumerate}

    Tras tantear los distintos frameworks para el desarrollo de aplicaciones web disponibles
    se ha decidido utilizar Django, ya que está hecho en un lenguaje familiar para mí y
    además cumple todos los requisitos anteriormente mencionados.

    Dicho framework permite hacer lo siguiente:
        
        \begin{itemize}
            \item Crear la aplicación sin tener que preocuparnos de elementos para que
            ésta funcione.
            \item Configuración con distintas bases de datos, SQLite es la base de datos
            por defecto, pero además permite configurarlo para trabajar con mySQL y
            postgreSQL.
            \item Sistema de autenticación de usuarios, por lo que podremos gestionar el
            registro e inicio de sesión de los mismos.
            \item Contiene un ORM que permite el manejo de modelos de la base de datos
            abstrayendo la base de datos utilizada.
            \item A pesar de no venir incluido como tal, instalando la aplicación Django
            REST framework, es posible realizar la configuración de la API de una forma
            relativamente sencilla.
            \item Tiene una documentación realmente buena, clara y sencilla.
        \end{itemize}
    
    Éstas son las razones por las que se ha decidido elegir Django para el desarrollo de
    nuestra aplicación web.

    \subsubsection{Elección del Framework CSS}
    Entre los requisitos que el framework CSS debe cumplir están los siguientes:

        \begin{enumerate}
            \item Que permita un diseño adaptativo de la página web.
            \item Que siga la metodología Responsive Web Design.
            \item Que sea fácil de usar.
            \item Que esté actualmente mantenido por la comunidad.
            \item Que posea una documentación simple y fácil de entender. 
        \end{enumerate}

    Siguiendo los requisitos de arriba, Foundation puede ser descartado ya que no cumple el
    segundo requisito, ya que nuestra aplicación web principalmente va a estar diseñada para
    utilizarse en un ordenador (también podrá usarse en un smartphone, pero no es la
    finalidad).

    Entre los tres frameworks restantes, podría utilizarse cualquiera de ellos ya que cumplen
    con los requisitos, pero finalmente se ha decidido utilizar Bootstrap ya que es
    probablemente el framework más popular y sencillo de usar que hay, además de seguir
    siendo muy usado por la comunidad. 
    

    \subsubsection{Elección de la Base de Datos}
    Para la elección de la Base de Datos que utilizará el servidor web para almacenar toda
    la información de las excavaciones se tienen que cumplir los siguientes requisitos:

        \begin{enumerate}
            \item Que sea una base de datos relacional.
            \item Que sea open-source y gratuita.
            \item Que tenga soporte para la realización de transacciones seguras.
            \item Que haga un uso eficiente de los datos.
        \end{enumerate}

    Por cuestiones de escalabilidad, utilizar SQLite se puede descartar ya que posee un
    reducido número de tipos de datos y más bien está pensada para utilizarse en dispositivos
    con poca capacidad de almacenamiento.

    Ahora nos tocaría pensar en, ¿mySQL o postgreSQL? Realmente podría utilizarse cualquiera
    de las dos, pero posgreSQL ofrece algunas ventajas respecto a mySQL que merece la pena
    mencionar: 
    
        \begin{itemize}
            \item MySQL es propiedad de Oracle y podría pasar a ser un producto comercial
            (de pago), sin embargo, postgreSQL es \textbf{open source 100\%}.
            \item PostgreSQL ofrece una mayor \textbf{integridad y fiabilidad} de los datos
            (principio ACID).
            \item Gracias a funciones que lectura y escritura en paralelo, postgreSQL ofrece
            una mayor \textbf{velocidad} frente a mySQL.
        \end{itemize}
    
    Por estas razones, finalmente se ha decidido usar postgreSQL como base de datos para
    nuestro servidor web.
