\chapter{Análisis del problema}
 
En todo proceso de Ingeniería del Software se necesita seguir un procedimiento muy metódico
y preciso, dividiendo el desarrollo del software en distintas fases que nos ayudarán a
conseguir un producto que se ajuste a las necesidades que requiere el problema.

\section{Ingeniería de requisitos}
Durante esta fase se cubrirán y proporcionarán las técnicas y los mecanismos apropiados
para:

    \begin{itemize}
        \item Analizar y entender las necesidades de los arqueólogos.
        \item Evaluar la viabilidad de las distintas propuestas (necesidades).
        \item Negociar un solución razonable/viable para ambas partes.
        \item Controlar y administrar los requisitos a lo largo del proceso de desarrollo.
    \end{itemize}

\subsection{Requisitos funcionales}
Describen la interacción entre la Aplicación Web y su entorno (base de datos, usuarios, 
comunicación con la App Android, etc), indicando cómo debe actuar
frente a situaciones o entradas determinadas.

    \begin{itemize}
        \item 
    \end{itemize}

\subsection{Requisitos no funcionales}
Este tipo de requisitos describen restricciones o cualidades de nuestra aplicación web que
no tienen una relación directa con las funcionalidades del sistema.

    \begin{itemize}
        \item 
    \end{itemize}

\subsection{Requisitos de información}
Este tipo de requisitos describirán necesidades de almacenamiento de información en
la aplicación web.

    \begin{itemize}
        \item 
    \end{itemize}
