\chapter{Descripción del problema}
Como hemos mencionado anteriormente, en arqueología se toman anotaciones mediante unas fichas
de registro en papel. Este tipo de metodología para la recogida de datos no es del todo cómoda
y práctica para el arqueólogo, ya que normalmente las herramientas de trabajo principales son
las manos, y escribir anotaciones se convierte en una tarea difícil de realizar. Además, no
podemos olvidar la complejidad que puede llegar a tener la gestión de todos los datos en
excavaciones complejas, que requieran de muchos registros en las fichas. \\

Con estos problemas en mente, como hemos citado anteriormente, surgió la petición expresa por
parte del arqueólogo, de la creación de un proyecto que les facilitara tanto la recogida de datos
de campo como la posterior incorporación de información en las \textbf{fichas de registro
ampliadas}. En este contexto, surgió la idea de este proyecto, una petición expresa por parte
del arqueólogo para la creación de una \textbf{aplicación web} con la que interactuaría una
aplicación Android. \\

Además de lo anteriormente mencionado, existen muchos otros problemas, como pueden ser los
siguientes:

    \begin{itemize}
        \item Gestión compleja para el inventario fotográfico.
        \item Difícil gestión de datos en papel cuando las excavaciones son complejas y
        existen gran cantidad de restos arqueológicos.
        \item Información de elementos no sincronizada entre arqueólogos que trabajan en la
        misma excavación.
        \item Elaboración costosa y lenta de la documentación necesaria de las excavaciones.
    \end{itemize}

Estos serían los principales inconvenientes que presentaría actualmente esta disciplina. En
el punto siguiente hablaremos de los objetivos que tendrá que cumplir nuestro proyecto para
poder solventarlos adecuadamente.

\section{Objetivos}
Durante el desarrollo de este proyecto, se deben ir cumpliendo de forma progresiva una serie
de objetivos que tengan como meta solucionar los problemas anteriormente mencionados. Los
objetivos a cumplir desde el punto de vista personal y técnico son los siguientes:

    \begin{enumerate}
        \item El principal objetivo de este proyecto es el diseño y creación de una
        \textbf{aplicación web} para toma y gestión de datos en excavaciones arqueológicas.
        Dentro de este gran objetivo, consideramos los siguientes:
            \begin{itemize}
                \item Crear un sitio web multiusuario, que permita a múltiples arqueólogos
                trabajar concurrentemente sobre una misma excavación y modificar sus datos
                simultáneamente.
                \item Garantizar la consistencia de los datos mediante su almacenamiento en
                un servidor web.
                \item Facilitar la documentación de excavaciones mediante la generación
                automática de informes.
                \item Creación de una interfaz simple e intuitiva para la gestión de
                excavaciones.
            \end{itemize}
            \item Otro punto que se irá adquiriendo de forma implícita será aprender las
            mejores prácticas en el ámbito de la programación y desarrollo web.
    \end{enumerate}

Estos serían los objetivos principales que queremos cumplir cuando finalicemos este proyecto. 
