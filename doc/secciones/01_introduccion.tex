\chapter{Introducción}
\section{Contexto}
En la actualidad, con el gran avance tecnológico que experimenta nuestra sociedad cada día,
ciertas disciplinas como la arqueología pasan en cierta medida desapercibidas, quedando en
un segundo plano para la sociedad. Sin embargo, no debemos olvidar su gran importancia en
el ámbito cultural y sociológico, pues gracias a \textbf{excavaciones} y \textbf{campañas
arqueológicas} se han descubierto misterios sin resolver de nuestra civilización y de
nuestros ancestros. Además, no podemos obviar el carácter turístico y económico que las
mismas poseen. \\

Dicho esto, para contextualizar nuestra investigación, vamos a definir brevemente en qué
consisten las excavaciones arqueológicas. Una \textbf{excavación arqueológica} consiste en
una intervención que se realiza sobre un terreno determinado con el objetivo de encontrar
restos de épocas anteriores. Normalmente, según la naturaleza de las mismas, nos encontramos
dos tipos \cite{excavation-type}:

    \begin{itemize}
        \item \textbf{Excavaciones planificadas}: son excavaciones arqueológicas que
        están basadas en proyectos científicos o de investigación. Normalmente, suelen
        tener una duración de un par de meses, en los que participan tanto arqueólogos
        profesionales como aprendices, como podrían ser alumnos de Historia.

        \item \textbf{Excavaciones de urgencia}: este tipo de excavaciones surgen por
        sorpresa mientras se está realizando una obra pública o privada. Normalmente,
        las obras se desplazan a otro punto o se detienen mientras los arqueólogos
        documentan los restos encontrados. Además, debido a que las obras destruirán el
        el yacimiento, será necesario la excavación total del mismo.\\
    \end{itemize}

Hoy en día, los arqueólogos utilizan para la documentación de las excavaciones fichas de
registro, que pueden ser tanto \textbf{reducidas} (contienen los datos mínimos necesarios)
como \textbf{ampliadas} (fichas más complejas que recogen todos los datos necesarios de una
excavación). La ficha reducida se emplea para el registro de campo, es decir, mientras se
está realizando la excavación, mientras que la ampliada, como su propio nombre indica, se
utiliza para la total documentación de la excavación. Por supuesto, una parte de la ficha
ampliada se rellena a partir de la reducida. \\ 

Otro de los aspectos más importantes a comentar es la notación usada para identificar los
distintos elementos. En arqueología, se sigue una notación muy estricta y bien definida para
los distintos restos arqueológicos que se van encontrando. Dicha notación define un estándar,
un lenguaje común con el que todos los que practican la disciplina pueden entenderse.

\section{Motivación}
En la sociedad actual, podemos observar que existe una fuente de desconocimiento con el mundo
de la arqueología y todo lo relacionado con la misma. Esto puede haber provocado que se sigan
usando métodos tradicionales para la recogida de datos, ralentizando que la misma pase por una
fase de digitalización. \\

Las razones anteriormente mencionadas han llevado a esta disciplina a un vacío tecnológico muy
destacado, siendo una de las evidencias más notables la escasez de aplicaciones o soluciones
software al respecto. Es por ello que, ante esta petición expresa por parte del arqueólogo
de la realización de este proyecto, compuesto por una \textbf{aplicación web} y Android, no se
dudó en iniciarlo. Nuestra aplicación podrá tomar una posición estratégica en el mercado por
ser pionera y ofrecer al cliente servicios nunca antes ofrecidos por otra aplicación.
