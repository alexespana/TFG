\chapter{Conclusiones y trabajos futuros}
Tras unos meses de duro trabajo, hemos conseguido implementar una aplicación web totalmente
funcional para los arqueólogos de la Universidad de Granada. A lo largo del proyecto, se han
ido teniendo reuniones periódicas con el tutor académico y con el arqueólogo, añadiendo
nuevas sugerencias y solucionando problemas que se han ido encontrando en la aplicación. De
esta forma, hemos conseguido realizar una aplicación a medida y gusto de nuestros clientes
potenciales. \\

Sin duda alguna, la realización de una aplicación web no es una tarea fácil, pues se deben
tener en cuenta muchas de las decisiones que se toman y las repercusiones que tendrán sobre
el proyecto, como por ejemplo, el uso de tecnologías modernas, el uso de herramientas de
desarrollo, el uso de lenguajes de programación, soluciones para poder ejecutar la aplicación
de manera sencilla en un ambiente de desarrollo, soluciones en la nube actuales para la
fase de despliegue, comprobación de la calidad del \textit{software} mediante test unitarios,
etc. \\

Pensando en mejoras para la aplicación, se contempla para futuros trabajos la
incorporación de nuevas funcionalidades como pueden ser las siguientes:

    \begin{itemize}
        \item Registo en la aplicación a través de otras aplicaciones como podrían ser
        \textbf{Facebook} o \textbf{Google}. Para ello, podría utilizarse el paquete
        \href{https://python-social-auth.readthedocs.io/en/latest/configuration/django.html}
        {social-auth-app-django}, que tiene soporte para ambos.

        \item Permitir exportar los informes de excavaciones en un mayor número de formatos,
        como podrían ser PDF, Microsoft Excel, HTML, etc.

        \item Encontrar un servicio de almacenamiento gratuito con más capacidad de
        almacenamiento que \href{https://cloudinary.com/}{Cloudinary}. Este servicio se
        emplea para las imágenes subidas por usuarios de la aplicación (normalmente
        referidas como \textbf{\textit{media files}}).

        \newpage \item Incorporación de mejoras en la aplicación a través de nuevas sugerencias
        que puedan llegar por parte de los arqueólogos durante su uso o de la comunidad, ya
        que se trata de un proyecto de código abierto (\textbf{\textit{open source}}) y
        distribuido bajo la licencia \href{https://www.gnu.org/licenses/gpl-3.0.html}{GNU
        General Public License v3.0}.
    \end{itemize}

Todos estos cambios pueden ir añadiéndose a la aplicación en los próximos meses, realizando
un nuevo despliegue de la misma que incluya dichas mejoras.
