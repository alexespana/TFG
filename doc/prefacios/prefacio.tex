\thispagestyle{empty}
\cleardoublepage
\begin{center}
{\large\bfseries Sistema de información para la toma y gestión de datos en excavaciones arqueológicas \\ Aplicación Web }\\
\end{center}
\begin{center}
	Joaquín Alejandro España Sánchez\\
\end{center}

%\vspace{0.7cm}

\vspace{0.5cm}
\noindent{\textbf{Palabras clave}: \textit{arqueología},            \textit{excavaciones}, 
								   \textit{generación de informes}, \textit{integración continua},
								   \textit{programación web},       \textit{Python},
								   \textit{servidor web},           \textit{software libre},
								   \textit{tests},                  \textit{toma de datos}}. \\
							     
\vspace{0.7cm}

\noindent{\textbf{Resumen}} \\

Actualmente, la toma y gestión de información en excavaciones arqueológicas se realiza sobre el papel, lo cual lo hace un trabajo poco práctico y difícil de realizar, tanto en el trabajo
de campo, es decir, durante la excavación, como en la posterior aportación más detallada de información. Para solucionar estos problemas, en este proyecto, se ha implementado una aplicación
web para la toma y gestión de datos en excavaciones arqueológicas. Esta aplicación recoge todo lo necesario para la gestión de datos en excavaciones, además de añadir funcionalidades
como la generación automática de informes en excavaciones y la gestión de usuarios, tanto para usuarios administradores del sistema como para usuarios autorizados. En el proceso, se han
ido empleando las mejores prácticas y técnicas de programación web, elaborando las distintas partes de la aplicación: diseño, base de datos, autenticación de usuarios, calidad del código
mediante tests, integración continua, creación de una API para que la comunicación con un cliente sea posible, registro de la aplicación, cambio a un ambiente de producción para el despliegue
de la aplicación, entornos de ejecución aislados para un ambiente de desarrollo, etc.


\clearpage

\begin{center}
	{\large\bfseries Information system for collecting and managing data in archaeological diggings\\ Web Aplication}\\
\end{center}
\begin{center}
	Joaquín Alejandro España Sánchez\\
\end{center}
\vspace{0.5cm}
\noindent{\textbf{Keywords}: \textit{archaeology},			\textit{excavations}, 
							 \textit{report generation},	\textit{continuous integration},
							 \textit{web programming},		\textit{Python},
							 \textit{web server}, 			\textit{open source},
							 \textit{tests},				\textit{data collection}}. \\
\vspace{0.7cm}

\noindent{\textbf{Abstract}} \\

Currently, the collection and management of information in archaeological excavations is done on paper, which makes it impractical and difficult to carry out, both in fieldwork, i.e. during the
excavation, and in the subsequent provision of more detailed information. To solve these problems, this project has implemented a web application for data collection and management in archaeological
excavations. This application includes everything necessary for the management of excavation data, as well as adding functionalities such as the automatic generation of excavation reports and user
management, both for system administrators and authorised users. In the process, the best practices and techniques of web programming have been used, elaborating the different parts of the application:
design, database, user authentication, code quality through tests, continuous integration, creation of an API to make communication with a client possible, application log, switching to a production
environment for application deployment, isolated execution environments for a development environment, etc.


\cleardoublepage

\thispagestyle{empty}

\noindent\rule[-1ex]{\textwidth}{2pt}\\[4.5ex]

D. \textbf{Daniel Sánchez Fernández}, profesor del Departamento de Ciencias de la Computación
e Inteligencia Artificial de la Universidad de Granada.

\vspace{0.5cm}

\textbf{Informo:}

\vspace{0.5cm}

Que el presente trabajo, titulado \textit{\textbf{Sistema de información para la toma y
gestión de datos en excavaciones arqueológicas}}, ha sido realizado bajo mi supervisión por
\textbf{Joaquín Alejandro España Sánchez}, y autorizo la defensa de dicho trabajo ante el
tribunal que corresponda.

\vspace{0.5cm}

Y para que conste, expiden y firman el presente informe en Granada a \today.

\vspace{1cm}

\textbf{El director:}

\vspace{5cm}

\noindent \textbf{Daniel Sánchez Fernández}

\chapter*{Agradecimientos}

A la \textbf{Universidad de Granada}, y más concretamente a la \textbf{Escuela
Técnica Superior de Ingenierías Informática y de Telecomunicación}, por haberme formado
durante estos cuatro años y darme todos los conocimientos necesarios para realizar este
proyecto. \\

A mi tutor académico, \textbf{Daniel Sánchez Fernández}, por ayudarme y guiarme durante
toda la elaboración del proyecto. \\

Al arqueólogo \textbf{Francisco Javier Brao Gonzalez}, por prestarse a colaborar
activamente en el proyecto, mostrando su punto de vista técnico sobre la disciplina. \\

A mi \textbf{familia}, por haber realizado el gran esfuerzo económico para que pueda
estudiar en Granada y en esta universidad, sin ellos nada de esto habría sido posible. \\

A mis \textbf{amigos}, por darme la motivación necesaria en aquellos momentos más
difíciles del proyecto. \\

A mi \textbf{novia}, por haberme aguantado en estos meses de duro trabajo y entender
que tenía que dedicarle más horas al proyecto que a ella. \\
